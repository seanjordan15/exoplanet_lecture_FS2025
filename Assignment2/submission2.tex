\documentclass{article}
\usepackage{fancyhdr}
\usepackage{xcolor}
\usepackage{hyperref}
\usepackage{siunitx}
\usepackage{graphicx}
\usepackage[outercaption]{sidecap}  


% --- FILL IN YOUR NAME AND NUMBER OF EXERCISE SHEET HERE ---
\newcommand{\thename}{Group XX}
\newcommand{\theteam}{Name1, Name2, Name3}
\newcommand{\theweek}{2}
\newcommand{\theproject}{1}
\newcommand{\thedate}{March 14, 2024}

\begin{document}

\pagestyle{fancy}
\fancyhf{}
\fancyhead[L]{Extrasolar Planets}
\fancyhead[C]{Plot Sketches - Week \theweek}
\fancyhead[R]{FS 2024}
\fancyfoot[L]{\thedate}
% \fancyfoot[R]{Page \thepage}

% \noindent \textit{Group name:} \thename \\
% \textit{Group members:} \theteam

\section*{Project \theproject \, - Part 3}


\begin{SCfigure}[][h!]
\centering
\includegraphics[width=0.6\textwidth]{30_projects/01_radial_velocity/data/ls_periodogram.pdf}
\caption{The caption should capture all information needed for a reader to understand and interpret the plot. It can also draw attention to particular details in the plot.}
\end{SCfigure}

\section*{Project \theproject \, - Part 4}

\begin{SCfigure}[][h!]
\centering
\includegraphics[width=0.6\textwidth]{30_projects/01_radial_velocity/data/exoplanet_mass_separation.pdf}
\caption{The caption should capture all information needed for a reader to understand and interpret the plot. It can also draw attention to particular details in the plot.}
\end{SCfigure}




\end{document}
